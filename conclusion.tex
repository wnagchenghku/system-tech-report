\section{Conclusion} \label{sec:conclusion}

\xxx is an RDMA-based \paxos system that can efficiently replicate general 
server programs. In this report, we have demonstrated the following properties 
of \xxx.

\textbf{Novelty.} \xxx's runtime system uses \emph{fast} RDMA primitives to 
invoke consensus processes on requests \emph{concurrently} 
(\S\ref{sec:concurrent}). Besides, it addresses several practical issues 
including \emph{atomic} delivery of RDMA messages (\S\ref{sec:atomic}) and  
\emph{transparent} replication. All these features make \xxx a novel RDMA-based 
\paxos protocol.

\textbf{Practicability.} Our initial results show that \xxx can provide fault 
tolerance to server programs with low performance overhead without requiring 
server developers' intervention (\S\ref{sec:evaluation}). Compared with \nprog 
server programs' unreplicated executions, \xxx incurred \latencyoverhead 
overhead in response time and \tputoverhead in throughput. Its consensus 
latency outperformed four traditional consensus protocols by at least 
\comptradlow and faster than a recent RDMA-based consensus protocol \dare by 
\fasterDARE in average. This consensus latency stayed almost constant to the 
number of replicas and concurrent requests. Furthermore, our integration of 
\xxx and virtual machine greatly improves the virtual machine's reliability 
(\S\ref{sec:vm-integration}), indicating \xxx has broad applications.

\textbf{Patent Protectability.} \xxx's carefully-designed consensus protocol 
uses fast RDMA primitives to invoke consensus processes concurrently and takes 
one round trip in the normal case, which is almost the most optimal performance 
in the context of RDMA-based \paxos protocols. Therefore, if a technique which 
also uses RDMA to build \paxos, it is nearly impossible for it be as fast 
and scalable as \xxx without leverages \xxx's mechanism.